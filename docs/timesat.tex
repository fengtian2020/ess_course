% Options for packages loaded elsewhere
\PassOptionsToPackage{unicode}{hyperref}
\PassOptionsToPackage{hyphens}{url}
%
\documentclass[
]{article}
\usepackage{amsmath,amssymb}
\usepackage{lmodern}
\usepackage{iftex}
\ifPDFTeX
  \usepackage[T1]{fontenc}
  \usepackage[utf8]{inputenc}
  \usepackage{textcomp} % provide euro and other symbols
\else % if luatex or xetex
  \usepackage{unicode-math}
  \defaultfontfeatures{Scale=MatchLowercase}
  \defaultfontfeatures[\rmfamily]{Ligatures=TeX,Scale=1}
\fi
% Use upquote if available, for straight quotes in verbatim environments
\IfFileExists{upquote.sty}{\usepackage{upquote}}{}
\IfFileExists{microtype.sty}{% use microtype if available
  \usepackage[]{microtype}
  \UseMicrotypeSet[protrusion]{basicmath} % disable protrusion for tt fonts
}{}
\makeatletter
\@ifundefined{KOMAClassName}{% if non-KOMA class
  \IfFileExists{parskip.sty}{%
    \usepackage{parskip}
  }{% else
    \setlength{\parindent}{0pt}
    \setlength{\parskip}{6pt plus 2pt minus 1pt}}
}{% if KOMA class
  \KOMAoptions{parskip=half}}
\makeatother
\usepackage{xcolor}
\IfFileExists{xurl.sty}{\usepackage{xurl}}{} % add URL line breaks if available
\IfFileExists{bookmark.sty}{\usepackage{bookmark}}{\usepackage{hyperref}}
\hypersetup{
  pdftitle={TIMESAT和MODIS物候分析实验指导},
  pdfauthor={田丰 武汉大学 遥感信息工程学院},
  hidelinks,
  pdfcreator={LaTeX via pandoc}}
\urlstyle{same} % disable monospaced font for URLs
\usepackage[margin=1in]{geometry}
\usepackage{color}
\usepackage{fancyvrb}
\newcommand{\VerbBar}{|}
\newcommand{\VERB}{\Verb[commandchars=\\\{\}]}
\DefineVerbatimEnvironment{Highlighting}{Verbatim}{commandchars=\\\{\}}
% Add ',fontsize=\small' for more characters per line
\usepackage{framed}
\definecolor{shadecolor}{RGB}{248,248,248}
\newenvironment{Shaded}{\begin{snugshade}}{\end{snugshade}}
\newcommand{\AlertTok}[1]{\textcolor[rgb]{0.94,0.16,0.16}{#1}}
\newcommand{\AnnotationTok}[1]{\textcolor[rgb]{0.56,0.35,0.01}{\textbf{\textit{#1}}}}
\newcommand{\AttributeTok}[1]{\textcolor[rgb]{0.77,0.63,0.00}{#1}}
\newcommand{\BaseNTok}[1]{\textcolor[rgb]{0.00,0.00,0.81}{#1}}
\newcommand{\BuiltInTok}[1]{#1}
\newcommand{\CharTok}[1]{\textcolor[rgb]{0.31,0.60,0.02}{#1}}
\newcommand{\CommentTok}[1]{\textcolor[rgb]{0.56,0.35,0.01}{\textit{#1}}}
\newcommand{\CommentVarTok}[1]{\textcolor[rgb]{0.56,0.35,0.01}{\textbf{\textit{#1}}}}
\newcommand{\ConstantTok}[1]{\textcolor[rgb]{0.00,0.00,0.00}{#1}}
\newcommand{\ControlFlowTok}[1]{\textcolor[rgb]{0.13,0.29,0.53}{\textbf{#1}}}
\newcommand{\DataTypeTok}[1]{\textcolor[rgb]{0.13,0.29,0.53}{#1}}
\newcommand{\DecValTok}[1]{\textcolor[rgb]{0.00,0.00,0.81}{#1}}
\newcommand{\DocumentationTok}[1]{\textcolor[rgb]{0.56,0.35,0.01}{\textbf{\textit{#1}}}}
\newcommand{\ErrorTok}[1]{\textcolor[rgb]{0.64,0.00,0.00}{\textbf{#1}}}
\newcommand{\ExtensionTok}[1]{#1}
\newcommand{\FloatTok}[1]{\textcolor[rgb]{0.00,0.00,0.81}{#1}}
\newcommand{\FunctionTok}[1]{\textcolor[rgb]{0.00,0.00,0.00}{#1}}
\newcommand{\ImportTok}[1]{#1}
\newcommand{\InformationTok}[1]{\textcolor[rgb]{0.56,0.35,0.01}{\textbf{\textit{#1}}}}
\newcommand{\KeywordTok}[1]{\textcolor[rgb]{0.13,0.29,0.53}{\textbf{#1}}}
\newcommand{\NormalTok}[1]{#1}
\newcommand{\OperatorTok}[1]{\textcolor[rgb]{0.81,0.36,0.00}{\textbf{#1}}}
\newcommand{\OtherTok}[1]{\textcolor[rgb]{0.56,0.35,0.01}{#1}}
\newcommand{\PreprocessorTok}[1]{\textcolor[rgb]{0.56,0.35,0.01}{\textit{#1}}}
\newcommand{\RegionMarkerTok}[1]{#1}
\newcommand{\SpecialCharTok}[1]{\textcolor[rgb]{0.00,0.00,0.00}{#1}}
\newcommand{\SpecialStringTok}[1]{\textcolor[rgb]{0.31,0.60,0.02}{#1}}
\newcommand{\StringTok}[1]{\textcolor[rgb]{0.31,0.60,0.02}{#1}}
\newcommand{\VariableTok}[1]{\textcolor[rgb]{0.00,0.00,0.00}{#1}}
\newcommand{\VerbatimStringTok}[1]{\textcolor[rgb]{0.31,0.60,0.02}{#1}}
\newcommand{\WarningTok}[1]{\textcolor[rgb]{0.56,0.35,0.01}{\textbf{\textit{#1}}}}
\usepackage{graphicx}
\makeatletter
\def\maxwidth{\ifdim\Gin@nat@width>\linewidth\linewidth\else\Gin@nat@width\fi}
\def\maxheight{\ifdim\Gin@nat@height>\textheight\textheight\else\Gin@nat@height\fi}
\makeatother
% Scale images if necessary, so that they will not overflow the page
% margins by default, and it is still possible to overwrite the defaults
% using explicit options in \includegraphics[width, height, ...]{}
\setkeys{Gin}{width=\maxwidth,height=\maxheight,keepaspectratio}
% Set default figure placement to htbp
\makeatletter
\def\fps@figure{htbp}
\makeatother
\setlength{\emergencystretch}{3em} % prevent overfull lines
\providecommand{\tightlist}{%
  \setlength{\itemsep}{0pt}\setlength{\parskip}{0pt}}
\setcounter{secnumdepth}{-\maxdimen} % remove section numbering
\ifLuaTeX
  \usepackage{selnolig}  % disable illegal ligatures
\fi

\title{TIMESAT和MODIS物候分析实验指导}
\author{\textbf{\href{http://jszy.whu.edu.cn/tian_feng/}{田丰} 武汉大学
遥感信息工程学院}}
\date{2022-03-27}

\begin{document}
\maketitle

\href{https://web.nateko.lu.se/timesat/timesat.asp}{TIMESAT软件}是由瑞典隆德大学的\href{https://web.nateko.lu.se/Personal/Lars.Eklundh/}{Lars
Eklundh}及瑞典马尔默大学的
\href{https://mau.se/en/persons/per.jonsson/}{Per
Jönsson}在大约20年前开发的遥感物候分析工具,被全球遥感界广泛使用,最近欧洲哥白尼计划项目下的高分辨率物候和生产力产品\href{https://land.copernicus.eu/pan-european/biophysical-parameters/high-resolution-vegetation-phenology-and-productivity}{HR-VPP}就是基于TIMESAT生成的。TIMESAT使用\href{https://baike.baidu.com/item/FORTRAN\%E8\%AF\%AD\%E8\%A8\%80/295590}{FORTRAN语言}编写,该语言面向科学计算,相较于其他编程语言,其运算效率号称是最快的。

本页介绍利用TIMESAT软件和MODIS
NDVI产品进行物候分析,包括:数据下载、数据准备、数据处理、结果制图、分析整个过程,详细英文教程可以参见\href{/}{TIMESAT用户手册}。

\hypertarget{ux6570ux636eux4e0bux8f7dux548cux9884ux5904ux7406}{%
\section{\texorpdfstring{\textbf{1.
数据下载和预处理}}{1. 数据下载和预处理}}\label{ux6570ux636eux4e0bux8f7dux548cux9884ux5904ux7406}}

\hypertarget{modis-ndvi-ux6807ux51c6ux4ea7ux54c1}{%
\section{1.1 MODIS NDVI
标准产品}\label{modis-ndvi-ux6807ux51c6ux4ea7ux54c1}}

\texttt{NDVI}是\texttt{MODIS}的标准产品之一,包括16天合成和月合成两种时间分辨率、250米/500米/1公里/0.05度四种空间分辨率,网址:\url{https://modis.gsfc.nasa.gov/data/dataprod/mod13.php},其中\texttt{MOD13}来自上午星\texttt{TERRA},\texttt{MYD13}来自下午星\texttt{AQUA}。

我们使用16天合成的1公里分辨率的 \texttt{MOD13A2\ v061}
版产品为例进行实验,其详细介绍在:\url{https://lpdaac.usgs.gov/products/mod13a2v061/}.

数据获取途径有很多,我们使用\href{https://lpdaacsvc.cr.usgs.gov/appeears/}{AppEEARS}

\hypertarget{appeears-ux6570ux636eux4e0bux8f7dux6d41ux7a0b}{%
\subsection{1.2 AppEEARS
数据下载流程}\label{appeears-ux6570ux636eux4e0bux8f7dux6d41ux7a0b}}

首先登录账户,点击菜单栏中的\texttt{Extract} →
\texttt{Area},然后选择\texttt{Start\ a\ new\ request}。

选择一个感兴趣区域,可以利用交互式地图选取,这里我们用方框选取从北方草原到南方森林的一个长条地区作为实验区,选择这里纯粹是个人爱好,你也可以选取其他任何区域,但是要注意:实验区的面积越大,你处理的数据量也就越大。给自己选定的区域一个名称,这里我们输入\texttt{modis\_course}。

指定影像数据的开始时间和结束时间,这里我们选择2011年到2021年。
在\texttt{Select\ the\ layers\ to\ include\ in\ the\ Sample}对话框中,输入\texttt{MOD13A2},找到\texttt{Terra\ MODIS\ Vegetation\ Indices\ (NDVI\ \&\ EVI)\ MOD13A2.061}并选择。在图层中选择\texttt{NDVI}和\texttt{pixel\_reliability}。

\texttt{pixel\_reliability}为数据质量概要信息,可以用来过滤掉质量差的数据点(去噪),下图为MOD13用户手册中的介绍,我们将使用这个数据作为TIMESAT的QA数据质量文件输入:

然后就可以点击提交了,之后从菜单栏的\texttt{Explore}里,可以看到你数据申请的处理状态,等到完成后(Done),就可以下载了,我们需要NDVI和pixel\_reliability两个数据。

如果下载过程因网络不稳定而中断了,可以利用搜索栏,选择未完成的文件继续下载。

\hypertarget{ux6570ux636eux9884ux5904ux7406}{%
\subsection{1.3 数据预处理}\label{ux6570ux636eux9884ux5904ux7406}}

下面我们将使用\texttt{R}编程,按照\texttt{TIMESAT}对输入数据的格式要求,对下载\texttt{MODIS}数据进行预处理。

\hypertarget{ux6587ux4ef6ux7ec4ux7ec7}{%
\subsubsection{文件组织}\label{ux6587ux4ef6ux7ec4ux7ec7}}

创建一个文件夹进行实验操作,其格式如下:

将MODIS
NDVI的tif文件拷贝到\texttt{..\textbackslash{}data\textbackslash{}GeoTIFF\textbackslash{}NDVI}中,质量标识文件
pixel
reliability的tif文件拷贝到\texttt{..\textbackslash{}data\textbackslash{}GeoTIFF\textbackslash{}QA}中。注意这两组数据应一一对应,以我们这个实验为例,NDVI和pixel
reliability分别有230个文件,从2011年到2021年,每年23个16天合成的数据。

\hypertarget{ux5f71ux50cfux6570ux636eux683cux5f0fux8f6cux6362}{%
\subsubsection{影像数据格式转换}\label{ux5f71ux50cfux6570ux636eux683cux5f0fux8f6cux6362}}

将下载的\texttt{GeoTIFF}数据转成二进制\texttt{ENVI}格式才能输入到\texttt{TIMESAT}中。\texttt{ENVI}软件的数据格式既是二进制+头文件(\texttt{*.hdr},对二进制文件的解释)。

这里,我们用\texttt{R}代码进行格式转换。首先,打开\texttt{RStudio},新建R
Script文件:

将以下代码粘贴到新建的\texttt{R\ Script}文件中,并保存在\texttt{Rcode}文件夹下。注意根据你的实际情况修改文件路径。

\begin{Shaded}
\begin{Highlighting}[]
\CommentTok{\#加载需要用的软件包}
\ControlFlowTok{if}\NormalTok{ (}\SpecialCharTok{!}\FunctionTok{require}\NormalTok{(}\StringTok{"tidyverse"}\NormalTok{)) }\FunctionTok{install.packages}\NormalTok{(}\StringTok{"tidyverse"}\NormalTok{); }\FunctionTok{library}\NormalTok{(tidyverse)}
\ControlFlowTok{if}\NormalTok{ (}\SpecialCharTok{!}\FunctionTok{require}\NormalTok{(}\StringTok{"terra"}\NormalTok{)) }\FunctionTok{install.packages}\NormalTok{(}\StringTok{"terra"}\NormalTok{); }\FunctionTok{library}\NormalTok{(terra)}

\CommentTok{\# 获取tif数据文件的路径列表}
\NormalTok{tif\_files }\OtherTok{\textless{}{-}} \FunctionTok{list.files}\NormalTok{(}\AttributeTok{path =} \StringTok{"D:/modis/data/GeoTIFF"}\NormalTok{, }
                        \AttributeTok{pattern =} \StringTok{".tif"}\NormalTok{,}
                        \AttributeTok{full.names =} \ConstantTok{TRUE}\NormalTok{,}
                        \AttributeTok{recursive =} \ConstantTok{TRUE}\NormalTok{)}

\CommentTok{\# 准备envi输出文件的路径列表}
\NormalTok{envi\_files }\OtherTok{\textless{}{-}}\NormalTok{ tif\_files }\SpecialCharTok{\%\textgreater{}\%} \FunctionTok{str\_replace}\NormalTok{(}\StringTok{".tif"}\NormalTok{, }\StringTok{".envi"}\NormalTok{) }\SpecialCharTok{\%\textgreater{}\%} 
  \FunctionTok{str\_replace}\NormalTok{(}\StringTok{"GeoTIFF"}\NormalTok{, }\StringTok{"ENVI"}\NormalTok{)}

\CommentTok{\# 读取tif文件为栅格数据}
\NormalTok{data }\OtherTok{\textless{}{-}} \FunctionTok{rast}\NormalTok{(tif\_files)}

\CommentTok{\# 生成envi格式的文件}
\FunctionTok{writeRaster}\NormalTok{(data, }\AttributeTok{filename =}\NormalTok{ envi\_files)}

\CommentTok{\# 准备envi格式数据列表的list文件,用于TIMESAT读取}
\NormalTok{num }\OtherTok{\textless{}{-}} \FunctionTok{length}\NormalTok{(envi\_files) }\SpecialCharTok{/} \DecValTok{2} \CommentTok{\# 影像时间序列个数}

\CommentTok{\# 将文件分隔符从“/” 转成 “\textbackslash{}”}
\NormalTok{envi\_files\_b }\OtherTok{\textless{}{-}} \FunctionTok{str\_replace\_all}\NormalTok{(envi\_files, }\StringTok{"/"}\NormalTok{, }\StringTok{"}\SpecialCharTok{\textbackslash{}\textbackslash{}\textbackslash{}\textbackslash{}}\StringTok{"}\NormalTok{) }

\CommentTok{\# 分别取出 NDVI 和 QA 文件路径列表}
\NormalTok{ndvi\_envi\_files\_b }\OtherTok{\textless{}{-}} \FunctionTok{str\_subset}\NormalTok{(envi\_files\_b, }\StringTok{"NDVI"}\NormalTok{)}
\NormalTok{qa\_envi\_files\_b }\OtherTok{\textless{}{-}} \FunctionTok{str\_subset}\NormalTok{(envi\_files\_b, }\StringTok{"pixel\_reliability"}\NormalTok{)}

\CommentTok{\#按照TIMESAT要求的格式组织NDVI list文件,并输出txt文件}
\FunctionTok{append}\NormalTok{(}\FunctionTok{as.character}\NormalTok{(num), ndvi\_envi\_files\_b) }\SpecialCharTok{\%\textgreater{}\%} 
  \FunctionTok{as.data.frame}\NormalTok{() }\SpecialCharTok{\%\textgreater{}\%} 
  \FunctionTok{write.table}\NormalTok{(}\AttributeTok{file =} \StringTok{"D:/modis/data/NDVI\_list.txt"}\NormalTok{,}
            \AttributeTok{sep =} \StringTok{"}\SpecialCharTok{\textbackslash{}n}\StringTok{"}\NormalTok{, }\AttributeTok{quote =} \ConstantTok{FALSE}\NormalTok{, }
            \AttributeTok{row.names =} \ConstantTok{FALSE}\NormalTok{, }\AttributeTok{col.names =} \ConstantTok{FALSE}\NormalTok{)}

\CommentTok{\#按照TIMESAT要求的格式组织QA list文件,并输出txt文件}
\FunctionTok{append}\NormalTok{(}\FunctionTok{as.character}\NormalTok{(num), qa\_envi\_files\_b) }\SpecialCharTok{\%\textgreater{}\%} 
  \FunctionTok{as.data.frame}\NormalTok{() }\SpecialCharTok{\%\textgreater{}\%} 
  \FunctionTok{write.table}\NormalTok{(}\AttributeTok{file =} \StringTok{"D:/modis/data/QA\_list.txt"}\NormalTok{,}
              \AttributeTok{sep =} \StringTok{"}\SpecialCharTok{\textbackslash{}n}\StringTok{"}\NormalTok{, }\AttributeTok{quote =} \ConstantTok{FALSE}\NormalTok{, }
              \AttributeTok{row.names =} \ConstantTok{FALSE}\NormalTok{, }\AttributeTok{col.names =} \ConstantTok{FALSE}\NormalTok{)}
\end{Highlighting}
\end{Shaded}

运行\texttt{R}文件的方法为:选中要运行的部分,点击右上角的\texttt{Run}图标,或\texttt{Ctrl\ +\ Enter}:

\hypertarget{ux8f6cux6362ux540eux7684ux6587ux4ef6}{%
\subsubsection{转换后的文件}\label{ux8f6cux6362ux540eux7684ux6587ux4ef6}}

转换完成后,\texttt{data/ENVI}文件夹中应该看到envi格式的NDVI和QA数据,如下图所示:

其中\texttt{.hdr}为头文件,包含了envi二进制数据的信息,可以用文本编辑器打开查看,如行列数、数据类型、投影等:

\texttt{data}文件夹中会生成两个\texttt{txt}文件,分别为\texttt{NDVI}和\texttt{pixel\_reliability}文件(即\texttt{TIMESAT}中要使用的\texttt{QA}文件)的路径列表(\texttt{list}),会作为\texttt{TIMESAT}软件的输入文件使用。\texttt{txt}文件的第一行为输入数据的总个数,这里是\texttt{230}个
= \texttt{23}个/年 * \texttt{10}年:

至此,数据准备完成,可以打开matlab运行TIMESAT软件了。

\hypertarget{ux8fd0ux884ctimesatux8ba1ux7b97ux7269ux5019ux53c2ux6570}{%
\section{\texorpdfstring{\textbf{2.
运行TIMESAT计算物候参数}}{2. 运行TIMESAT计算物候参数}}\label{ux8fd0ux884ctimesatux8ba1ux7b97ux7269ux5019ux53c2ux6570}}

\hypertarget{matlabux8bbeux7f6e}{%
\subsection{2.1 matlab设置}\label{matlabux8bbeux7f6e}}

将实验文件夹设为工作目录,这里是\texttt{D:\textbackslash{}modis}。然后点击\texttt{Set\ Path}添加\texttt{TIMESAT}软件的路径:

选择\texttt{Add\ with\ Subfolders..},导航到\texttt{timesat33}软件的路径,选择\texttt{timesat\_matlab}文件夹:

然后在\texttt{matlab}命令行输入\texttt{timesat},回车,就可以打开软件界面啦,包含三个模块:

\hypertarget{timesatux67e5ux770bux6570ux636eux548cux751fux6210ux8bbeux7f6eux6587ux4ef6}{%
\subsection{2.2
TIMESAT查看数据和生成设置文件}\label{timesatux67e5ux770bux6570ux636eux548cux751fux6210ux8bbeux7f6eux6587ux4ef6}}

点击\texttt{TSM\_imageview},打开之后选择\texttt{File\ →\ open\ file\ list}:

选择之前在\texttt{R}中生成的\texttt{list}文件\texttt{..data/NDVI\_list.txt},在\texttt{TSM\_imageview}窗口中输入影像的数据类型、行列数,这些信息可以打开任一个\texttt{ENVI}的头文件\texttt{.hdr}获取,填好以后,点击\texttt{Draw},就可以看到实验区\texttt{NDVI}的空间分布了,在列表中选择任一个时间的文件,点击\texttt{Draw}之后就会更新:

点击\texttt{TSM\_GUI},打开之后选择\texttt{File\ →\ Open\ list\ image\ files}:

选择之前在\texttt{R}中生成的\texttt{NDVI}和\texttt{QA}的\texttt{list}文件,配置如下所示:

其中,\texttt{QA}用来根据不同数据质量赋予\texttt{NDVI}观测点权重,这里我们把天空晴朗的无云观测(\texttt{QA}值为\texttt{0})赋予权重为\texttt{1};把数据质量一般的观测(\texttt{QA}值为\texttt{1})赋予权重为\texttt{0.5},有云和雪覆盖的观测(\texttt{QA}值为\texttt{2-3})赋予权重为\texttt{0}。

点击\texttt{Show\ image}可以选择显示时间序列的窗口范围:\texttt{Rows\ to\ process}和\texttt{Columns\ to\ process}:

选择完成之后,点击\texttt{Load\ data}就可以看到时间序列啦:

这里是查看时间序列、选定拟合函数、以及设置各种物候提取参数的界面,尝试点击调整不同的设置,探索最佳的或可以接受的参数设置,然后选择\texttt{Settings\ →\ Save\ to\ settings\ file}
来设置\texttt{settings}文件:

检查一下设置,然后就可以保存设置文件到\texttt{/run}文件夹了。这里可以为每一种植被类型(\texttt{Land\ cover})设置不同的参数,我们暂且用同一种参数实验。准备好\texttt{settings}文件后,就可以运行处理了。

\hypertarget{timesatux6267ux884cux6570ux636eux5904ux7406}{%
\subsection{2.3
TIMESAT执行数据处理}\label{timesatux6267ux884cux6570ux636eux5904ux7406}}

\hypertarget{ux5355ux6838ux5904ux7406}{%
\subsubsection{单核处理}\label{ux5355ux6838ux5904ux7406}}

点击\texttt{TSF\_process},选择配置文件\texttt{modis.set}:

桌面会弹出\texttt{cmd}窗口,\texttt{TIMESAT}开始逐行处理影像时间序列:

\hypertarget{ux5e76ux884cux5904ux7406}{%
\subsubsection{并行处理}\label{ux5e76ux884cux5904ux7406}}

如果数据处理量较大,\texttt{TIMESAT}还提供了并行处理。可以按\texttt{Ctrl\ +\ c}停止,然后点击\texttt{TSF\_process\ parallel},键入想要使用的处理器个数(最好不要把所有的\texttt{CUP}都占用,容易死机):

然后,会弹出\texttt{cmd}窗口,同时\texttt{TIMESAT}会将生成相应个数的\texttt{settings}文件,以及\texttt{.bat}批处理文件。在\texttt{cmd}窗口中键入
\texttt{.bat}文件名,\texttt{TIMESAT}就开始进行并行处理了:

处理完成后,\texttt{/run}文件夹中会生成\texttt{.tts}、\texttt{.tpa}、\texttt{.ndx}文件,这是TIMESAT的中间文件,可以用来查看处理结果,经过后处理即可生成\texttt{ENVI}二进制格式的影像结果了。

\hypertarget{timesatux7ed3ux679cux67e5ux770bux548cux540eux5904ux7406}{%
\section{\texorpdfstring{\textbf{3
TIMESAT结果查看和后处理}}{3 TIMESAT结果查看和后处理}}\label{timesatux7ed3ux679cux67e5ux770bux548cux540eux5904ux7406}}

\hypertarget{ux751fux6210ux62dfux5408ux65f6ux95f4ux5e8fux5217ux5f71ux50cf}{%
\subsection{3.1
生成拟合时间序列影像}\label{ux751fux6210ux62dfux5408ux65f6ux95f4ux5e8fux5217ux5f71ux50cf}}

曲线拟合可以将原始数据中存在云污染等的观测值进行填补,生成时空连续的影像序列,可以用\texttt{TSM\_viewfits}来查看拟合结果:

\texttt{TIMESAT}输出的拟合结果去除了噪音,具有时空连续性,可以将该结果输出为二进制影像文件,点击\texttt{TSF\_fit2img}:

按照提示在\texttt{cmd}窗口中键入参数,生成结果:

现在\texttt{run}文件夹中的文件很多了,为了更清楚的组织文件,我们可以新建文件夹\texttt{fitted\_image/TIMESATout},并把生成拟合影像序列移动过去。

为了查看拟合后的时空连续的\texttt{NDVI}影像,我们介绍两种方法:

\begin{enumerate}
\def\labelenumi{\arabic{enumi}.}
\tightlist
\item
  利用如下\texttt{R}代码生成\texttt{list}文件\texttt{txt},就可以使用\texttt{TIMESAT}的\texttt{TSM\_imageview}查看拟合后的影像(记得根据自己实际情况更改文件路径):
\end{enumerate}

\begin{Shaded}
\begin{Highlighting}[]
\CommentTok{\# 生成TIMESAT读list需要的txt文件}

\CommentTok{\#加载需要用的软件包}
\ControlFlowTok{if}\NormalTok{ (}\SpecialCharTok{!}\FunctionTok{require}\NormalTok{(}\StringTok{"tidyverse"}\NormalTok{)) }\FunctionTok{install.packages}\NormalTok{(}\StringTok{"tidyverse"}\NormalTok{); }\FunctionTok{library}\NormalTok{(tidyverse)}

\CommentTok{\# 获取影像数据文件的路径列表}
\NormalTok{files }\OtherTok{\textless{}{-}} \FunctionTok{list.files}\NormalTok{(}\AttributeTok{path =} \StringTok{"D:/modis/run/fitted\_image/TIMESATout"}\NormalTok{, }
                    \AttributeTok{pattern =} \StringTok{"modis\_fited\_image\_"}\NormalTok{,}
                    \AttributeTok{full.names =} \ConstantTok{TRUE}\NormalTok{) }\SpecialCharTok{\%\textgreater{}\%} 
  \FunctionTok{str\_subset}\NormalTok{(}\StringTok{".hdr"}\NormalTok{, }\AttributeTok{negate =}\NormalTok{ T)}

\CommentTok{\# 准备数据列表的list文件,用于TIMESAT读取}
\NormalTok{num }\OtherTok{\textless{}{-}} \FunctionTok{length}\NormalTok{(files)  }\CommentTok{\# 影像时间序列个数}
\NormalTok{files\_b }\OtherTok{\textless{}{-}} \FunctionTok{str\_replace\_all}\NormalTok{(files, }\StringTok{"/"}\NormalTok{, }\StringTok{"}\SpecialCharTok{\textbackslash{}\textbackslash{}\textbackslash{}\textbackslash{}}\StringTok{"}\NormalTok{) }\CommentTok{\# 将文件分隔符从“/” 转成 “\textbackslash{}”}

\CommentTok{\#按照TIMESAT要求的格式组织NDVI list文件,并输出txt文件}
\FunctionTok{append}\NormalTok{(}\FunctionTok{as.character}\NormalTok{(num), files\_b) }\SpecialCharTok{\%\textgreater{}\%} 
  \FunctionTok{as.data.frame}\NormalTok{() }\SpecialCharTok{\%\textgreater{}\%} 
  \FunctionTok{write.table}\NormalTok{(}\AttributeTok{file =} \StringTok{"D:/modis/run/modis\_fited\_image\_list.txt"}\NormalTok{,}
              \AttributeTok{sep =} \StringTok{"}\SpecialCharTok{\textbackslash{}n}\StringTok{"}\NormalTok{, }\AttributeTok{quote =} \ConstantTok{FALSE}\NormalTok{, }
              \AttributeTok{row.names =} \ConstantTok{FALSE}\NormalTok{, }\AttributeTok{col.names =} \ConstantTok{FALSE}\NormalTok{)}
\end{Highlighting}
\end{Shaded}

\begin{enumerate}
\def\labelenumi{\arabic{enumi}.}
\setcounter{enumi}{1}
\tightlist
\item
  给\texttt{TIMESAT}生成的拟合二进制文件添加\texttt{.hdr}头文件,并转成\texttt{tif},这样就可以用其他软件(比如ENVI和GIS软件等)进行查看了。首先在\texttt{fitted\_image}文件夹下创建\texttt{tif}文件夹,然后运行下面的\texttt{R}代码:
\end{enumerate}

\begin{Shaded}
\begin{Highlighting}[]
\CommentTok{\# 将TIMESAT生成的二进制文件图像转换成tif格式}

\CommentTok{\#加载需要用的软件包}
\ControlFlowTok{if}\NormalTok{ (}\SpecialCharTok{!}\FunctionTok{require}\NormalTok{(}\StringTok{"tidyverse"}\NormalTok{)) }\FunctionTok{install.packages}\NormalTok{(}\StringTok{"tidyverse"}\NormalTok{); }\FunctionTok{library}\NormalTok{(tidyverse)}
\ControlFlowTok{if}\NormalTok{ (}\SpecialCharTok{!}\FunctionTok{require}\NormalTok{(}\StringTok{"terra"}\NormalTok{)) }\FunctionTok{install.packages}\NormalTok{(}\StringTok{"terra"}\NormalTok{); }\FunctionTok{library}\NormalTok{(terra)}

\CommentTok{\# 获取影像数据文件的路径列表}
\NormalTok{files }\OtherTok{\textless{}{-}} \FunctionTok{list.files}\NormalTok{(}\AttributeTok{path =} \StringTok{"D:/modis/run/fitted\_image/TIMESATout"}\NormalTok{,}
                    \AttributeTok{pattern =} \StringTok{"modis\_fited\_image\_"}\NormalTok{,}
                    \AttributeTok{full.names =} \ConstantTok{TRUE}\NormalTok{) }\SpecialCharTok{\%\textgreater{}\%}
  \FunctionTok{str\_subset}\NormalTok{(}\StringTok{".hdr"}\NormalTok{, }\AttributeTok{negate =}\NormalTok{ T)}

\CommentTok{\# 生成头文件 {-}{-}{-}{-}{-}{-}{-}{-}{-}{-}{-}{-}{-}{-}{-}{-}{-}{-}{-}{-}{-}{-}{-}{-}{-}{-}{-}{-}{-}{-}{-}{-}{-}{-}{-}{-}{-}{-}{-}{-}{-}{-}{-}{-}{-}{-}{-}{-}{-}{-}{-}{-}{-}{-}{-}{-}{-}{-}{-}{-}{-}{-}{-}{-}{-}{-}{-}}

\NormalTok{outname }\OtherTok{\textless{}{-}} \FunctionTok{paste0}\NormalTok{(files, }\StringTok{".hdr"}\NormalTok{)}

\CommentTok{\# 读取原始NDVI数据的头文件}
\NormalTok{hdr }\OtherTok{\textless{}{-}} \FunctionTok{read\_file}\NormalTok{(}\StringTok{"D:/modis/run/add.hdr"}\NormalTok{)}

\CommentTok{\# 生成所有拟合影像文件相应的hdr文件}
\ControlFlowTok{for}\NormalTok{ (i }\ControlFlowTok{in} \FunctionTok{c}\NormalTok{(}\DecValTok{1}\SpecialCharTok{:} \FunctionTok{length}\NormalTok{(files))) \{}
  \FunctionTok{write\_file}\NormalTok{(hdr, outname[i])}
\NormalTok{\}}

\CommentTok{\# 输出文件命名,对应到原始NDVI影像的命名方式}
\NormalTok{doy\_file\_names }\OtherTok{\textless{}{-}} \FunctionTok{list.files}\NormalTok{(}\StringTok{"D:/modis/data/GeoTIFF/NDVI"}\NormalTok{, }\AttributeTok{pattern =} \StringTok{".tif"}\NormalTok{) }\SpecialCharTok{\%\textgreater{}\%} 
  \FunctionTok{str\_extract}\NormalTok{(}\StringTok{"doy}\SpecialCharTok{\textbackslash{}\textbackslash{}}\StringTok{d\{7\}"}\NormalTok{) }\SpecialCharTok{\%\textgreater{}\%} \FunctionTok{paste0}\NormalTok{(}\StringTok{"\_fitted.tif"}\NormalTok{)}

\NormalTok{tif\_out\_files }\OtherTok{\textless{}{-}} \FunctionTok{paste0}\NormalTok{(}\StringTok{"D:/modis/run/fitted\_image/tif/"}\NormalTok{, doy\_file\_names)}

\CommentTok{\# 生成拟合后的TIFF格式影像}
\FunctionTok{writeRaster}\NormalTok{(}\FunctionTok{rast}\NormalTok{(files), tif\_out\_files, }\AttributeTok{overwrite =}\NormalTok{ T)}
\end{Highlighting}
\end{Shaded}

\hypertarget{ux751fux6210ux7269ux5019ux53c2ux6570ux7ed3ux679cux5f71ux50cf}{%
\subsection{3.2
生成物候参数结果影像}\label{ux751fux6210ux7269ux5019ux53c2ux6570ux7ed3ux679cux5f71ux50cf}}

\hypertarget{ux5355ux4e2aux7269ux5019ux53c2ux6570ux5355ux4e2aux5e74ux4efdux5f15ux5bfcux7684ux65b9ux5f0f}{%
\subsubsection{单个物候参数、单个年份引导的方式:}\label{ux5355ux4e2aux7269ux5019ux53c2ux6570ux5355ux4e2aux5e74ux4efdux5f15ux5bfcux7684ux65b9ux5f0f}}

点击\texttt{TSF\_seas2img}生成物候参数影像:

按照提示输入参数:

使用\texttt{TSM\_imageview}查看物候参数:

\hypertarget{ux5229ux7528.batux6587ux4ef6ux8fdbux884cux6279ux91cfux751fux6210}{%
\subsubsection{利用.bat文件进行批量生成}\label{ux5229ux7528.batux6587ux4ef6ux8fdbux884cux6279ux91cfux751fux6210}}

\textbf{利用上面的步骤可以检查和理解TIMESAT物候参数的输出格式和设计思路:}

\begin{enumerate}
\def\labelenumi{\arabic{enumi}.}
\item
  每年最多可以探测出两个生长季,如果每一个像元只有一个生长季,则其第二个季节的物候参数为填充值
\item
  对于时间类的输出结果,如生长季开始时间\texttt{start-of-season\ time}
  (\texttt{SOS}),其结果为整个影像时间序列的序号,而非我们常用的日期或\texttt{DOY}
  (Day of Year,一年中的天数)。
\end{enumerate}

如果要生成多个物候参数、多年的结果,用前面的方式则太过繁琐,对此,我们可以在命令行中一次性的输入上述引导步骤中所需要的所有参数,例如,上图\texttt{cmd}中的过程就可以用一行代码来实现:

\begin{Shaded}
\begin{Highlighting}[]
\ExtensionTok{D:\textbackslash{}modis\textbackslash{}run}\OperatorTok{\textgreater{}}\NormalTok{ TSF\_seas2img modis\_TS.tpa 1 1 23 }\AttributeTok{{-}1} \AttributeTok{{-}2}\NormalTok{ sos\_2012 3}
\end{Highlighting}
\end{Shaded}

其中,TSF\_seas2img为TIMESAT中的函数,modis\_TS.tpa为之前运行TIMESAT计算物候参数后生成的文件,后面数字1表示具体物候参数指标(见下图),之后的数字参数含义也都可以参看上图\texttt{cmd}中引导对话框的内容。

因此,每个物候参数、每一年的提取都可以用一行代码来实现,把这些代码穿起来组织成一个\texttt{.bat}文件,就可以像之前进行\texttt{TIMESAT}数据并行处理(\texttt{TSF\_process\ parallel})那样,利用\texttt{.bat}文件进行物候参数提取的批处理了。我们用下面的\texttt{R}代码来生成这个\texttt{.bat}文件(生成的\texttt{.bat}文件可以用文本编辑器打开,见代码之后的图):

\begin{Shaded}
\begin{Highlighting}[]
\CommentTok{\# 生成批处理的.bat文件,将TIMESAT的物候参数输出为影像}

\ControlFlowTok{if}\NormalTok{ (}\SpecialCharTok{!}\FunctionTok{require}\NormalTok{(}\StringTok{"tidyverse"}\NormalTok{)) }\FunctionTok{install.packages}\NormalTok{(}\StringTok{"tidyverse"}\NormalTok{); }\FunctionTok{library}\NormalTok{(tidyverse)}

\NormalTok{cmd }\OtherTok{\textless{}{-}} \ConstantTok{NULL}
\CommentTok{\# 对所有物候参数进行循环}
\ControlFlowTok{for}\NormalTok{ (para }\ControlFlowTok{in} \FunctionTok{c}\NormalTok{(}\DecValTok{1}\NormalTok{,}\DecValTok{2}\NormalTok{,}\DecValTok{3}\NormalTok{,}\DecValTok{5}\NormalTok{,}\DecValTok{10}\NormalTok{))\{ }\CommentTok{\# 将准备提取的物候参数对应的数字填到这里}
  
  \CommentTok{\# 建立物候参数名称和代表值的查找表,对应TIMESAT的物候参数指标}
\NormalTok{  pheno }\OtherTok{\textless{}{-}} \FunctionTok{case\_when}\NormalTok{(}
\NormalTok{              para }\SpecialCharTok{==} \DecValTok{1} \SpecialCharTok{\textasciitilde{}} \StringTok{"sos"}\NormalTok{,       }\CommentTok{\# Start{-}of{-}season time}
\NormalTok{              para }\SpecialCharTok{==} \DecValTok{2} \SpecialCharTok{\textasciitilde{}} \StringTok{"eos"}\NormalTok{,       }\CommentTok{\# End{-}of{-}season time}
\NormalTok{              para }\SpecialCharTok{==} \DecValTok{3} \SpecialCharTok{\textasciitilde{}} \StringTok{"los"}\NormalTok{,       }\CommentTok{\# Length of season}
\NormalTok{              para }\SpecialCharTok{==} \DecValTok{4} \SpecialCharTok{\textasciitilde{}} \StringTok{"base"}\NormalTok{,      }\CommentTok{\# Base value}
\NormalTok{              para }\SpecialCharTok{==} \DecValTok{5} \SpecialCharTok{\textasciitilde{}} \StringTok{"middle"}\NormalTok{,    }\CommentTok{\# Time of middle of season}
\NormalTok{              para }\SpecialCharTok{==} \DecValTok{6} \SpecialCharTok{\textasciitilde{}} \StringTok{"maxval"}\NormalTok{,    }\CommentTok{\# Maximum value value of fitted data}
\NormalTok{              para }\SpecialCharTok{==} \DecValTok{7} \SpecialCharTok{\textasciitilde{}} \StringTok{"amp"}\NormalTok{,       }\CommentTok{\# Amplitude}
\NormalTok{              para }\SpecialCharTok{==} \DecValTok{8} \SpecialCharTok{\textasciitilde{}} \StringTok{"lder"}\NormalTok{,    }\CommentTok{\# Left derivative}
\NormalTok{              para }\SpecialCharTok{==} \DecValTok{9} \SpecialCharTok{\textasciitilde{}} \StringTok{"rder"}\NormalTok{,    }\CommentTok{\# Right derivative}
\NormalTok{              para }\SpecialCharTok{==} \DecValTok{10} \SpecialCharTok{\textasciitilde{}} \StringTok{"integral"}\NormalTok{, }\CommentTok{\# Large integral}
\NormalTok{              para }\SpecialCharTok{==} \DecValTok{11} \SpecialCharTok{\textasciitilde{}} \StringTok{"sintegral"}\NormalTok{,    }\CommentTok{\# Small integral}
\NormalTok{              para }\SpecialCharTok{==} \DecValTok{12} \SpecialCharTok{\textasciitilde{}} \StringTok{"sosval"}\NormalTok{,    }\CommentTok{\# Start{-}of{-}season value}
\NormalTok{              para }\SpecialCharTok{==} \DecValTok{13} \SpecialCharTok{\textasciitilde{}} \StringTok{"eosval"}\NormalTok{,    }\CommentTok{\# End{-}of{-}season value}
\NormalTok{            )}
  
    \CommentTok{\# 对所有年分循环}
  \ControlFlowTok{for}\NormalTok{ (year }\ControlFlowTok{in} \FunctionTok{c}\NormalTok{(}\DecValTok{2012}\SpecialCharTok{:}\DecValTok{2021}\NormalTok{)) \{   }\CommentTok{\# 准备提取的年份}
    
    \CommentTok{\# 找出每年23个影像在整个时间序列的位置}
\NormalTok{    idx\_start }\OtherTok{\textless{}{-}}\NormalTok{ (year }\SpecialCharTok{{-}} \DecValTok{2012}\NormalTok{) }\SpecialCharTok{*} \DecValTok{23} \SpecialCharTok{+} \DecValTok{1}
\NormalTok{    idx\_end }\OtherTok{\textless{}{-}}\NormalTok{ (year }\SpecialCharTok{{-}} \DecValTok{2012}\NormalTok{) }\SpecialCharTok{*} \DecValTok{23} \SpecialCharTok{+} \DecValTok{1} \SpecialCharTok{+} \DecValTok{23}
\NormalTok{    outname }\OtherTok{\textless{}{-}} \FunctionTok{paste}\NormalTok{(pheno, year, }\AttributeTok{sep =} \StringTok{"\_"}\NormalTok{)}
    
    \CommentTok{\# 命令行代码}
\NormalTok{    tmp }\OtherTok{\textless{}{-}} \FunctionTok{paste}\NormalTok{(}\StringTok{"TSF\_seas2img"}\NormalTok{, }\StringTok{"modis\_TS.tpa"}\NormalTok{, para, idx\_start, idx\_end, }\SpecialCharTok{{-}}\DecValTok{1}\NormalTok{, }\SpecialCharTok{{-}}\DecValTok{2}\NormalTok{, outname, }\DecValTok{3}\NormalTok{, }\AttributeTok{sep =} \StringTok{" "}\NormalTok{)}
    
\NormalTok{    cmd }\OtherTok{\textless{}{-}} \FunctionTok{c}\NormalTok{(cmd, tmp)}
\NormalTok{  \}}
\NormalTok{\}}

\FunctionTok{write\_lines}\NormalTok{(cmd, }\StringTok{"D:/modis/run/season2img.bat"}\NormalTok{)}
\end{Highlighting}
\end{Shaded}

之后就可以在\texttt{cmd}命令行中输入\texttt{.bat}文件名进行批处理了:

\begin{Shaded}
\begin{Highlighting}[]
\ExtensionTok{D:\textbackslash{}modis\textbackslash{}run}\OperatorTok{\textgreater{}}\NormalTok{season2img.bat}
\end{Highlighting}
\end{Shaded}

运行完成后,你的文件夹中就生成了物候参数的二进制文件,这些二进制影像文件可以使用TIMESAT的\texttt{TSM\_imageview}工具来查看了。

\hypertarget{ux7269ux5019ux53c2ux6570ux6570ux636eux540eux5904ux7406}{%
\subsection{3.3
物候参数数据后处理}\label{ux7269ux5019ux53c2ux6570ux6570ux636eux540eux5904ux7406}}

首先,我们将提取的物候参数结果移动到一个单独的文件夹中,创建\texttt{pheno\_img/TIMESATout}文件夹和子文件夹,将结果移动过来,然后在pheno\_img文件夹下创建\texttt{tif}文件夹用来存放tif文件

\texttt{TIMESAT}的\texttt{SOS}和\texttt{EOS}的输出结果是整个时间序列的序号,为此,我们要通过计算将其转换成\texttt{DOY}或是具体的日期。这个步骤看似容易,但实际比较棘手(tricky),因为有的地方一个完整的生长季可以是跨年的,比如南方的生长季结束时间有可能出现在第二年的2月份。对此,我们将以生长季峰值(\texttt{Time\ of\ middle\ of\ season})出现的年份为准,来转换生长季开始(\texttt{SOS})和结束(\texttt{EOS})的时间。

\end{document}
